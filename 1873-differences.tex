\documentclass[a4paper,twocolumn]{article}

\usepackage{fullpage}
\usepackage[compact]{titlesec}
\usepackage[hidelinks]{hyperref}
\usepackage[all]{hypcap}

\title{Harzbahn 1873 Differences}
\author{Designed by Ian D Wilson}
\date{Summary by Xeryus Stokkel}

\begin{document}

\maketitle

\section{Map}
\begin{itemize}
	\item Thick black lines are state railway tracks, companies don't use these.
	\item Thin black lines are concession routes.
	\item White and black dashed lines are narrow gauge track, companies do use
	these.
	\item Hexes with red borders are towns, all other hexes with dots or circles
	are villages. Hexes with two separate circles on them are still considered
	one village.
	\item White and pink circles are known as industries, numbered industries
	are mines.
\end{itemize}

\section{Game start}
\begin{itemize}
	\item The bank is of unlimited size.
	\item The start player is also in charge of the bank.
	\item Evenly distribute 4200M between all players as start capital.
\end{itemize}

\subsection{Start auction round}
In the initial auction round only the 15 mine certificates, the HBE and the GHE
are available. Start with the Start Premium bid, players bid consecutively
starting with 0M in increases of 10M. Players must either raise or pass, players
who have passed are out of the auction. Turn order is redetermined in reverse
order of passing (the winner goes first, the first to pass last). The player who
has won the bid \emph{must} buy one of the mines or concessions for face value
or minimum bid plus the Start Premium to the bank.

Next the auction continues as follows. A player may pass or buy one of the
remaining mines or concessions for face value or minimum bid plus the Start
Premium. When all players have passed the Start Premium is lowered by 10M and
another round of auctioning is started with the new Start Premium. The auction
ends when all mines and concessions have been bought or the Start Premium has
dropped to 0M and all players have passed. All unsold mines are closed, their
certificates are returned to the bank and their markers are turned to their
inactive side. Concessions are available again in the next AR.

After the Start Auction Round a regular SR is started. Turn order is the same
as the second part of the Start Auction Round.

\section{Trading Rounds}
The TRs consist of an Auction Round (AR) and a Stock Round (SR).

\subsection{Auction Rounds}
During ARs only the concessions are available, private mining companies are only
sold during the Start Auction Round. Which concessions are available depends on
the game phase, see table 2 in the rules. During an AR companies in receivership
are also available for auction.

The AR is round robin, a player may on his turn start an auction for a
concession or the option to purchase shares of a company in receivership. The
minimum bid is 100M and a bid must always be a multiple of 10M. Players may
increase the bid in with multiples of 10M with a mimimum of 10M. Players who
pass are out of the current auction. Once all players have passed the player
with the highest bid wins and pays the bid amount to the bank. Now the next
player has the opportunity to auction an item if available. If there are no
more items to auction then the AR is over. If all players pass on the
opportunity to start an auction then the AR is also over, unauctioned items are
available again during the next AR. If there are no items to auction then the
AR is skipped.

\subsection{Stock Rounds}
A player can sell one or more shares in all SRs except in the first SR. After
that he may do one of the following: buy a share, start a railway company, start
a public mining company, or exercise a purchase option. If the player does not
want to sell shares and perform an action he may pass.

\subsection{Shares and directorship}
Companies can have the following share splits:
\begin{itemize}
	\item Two 50\% shares (public mining companies only).
	\item Five 20\% shares.
	\item Ten 10\% shares.
\end{itemize}

The value of a share in a company is always the stock value, regardless of the
split. A company that has been formed and that is not in receivership always has
a director. The directorship can change when a player has more share in the
company than its current director. Exception: the director of a railway company
remains its director when he owns the associated concession and the company has
not bought its compulsory train yet, or it becomes insolvent.

\subsection{Selling shares}
Shares are sold in blocks for a single company. Sold shares end up in the bank
pool. The player receives the current stock value per share sold from the bank.
Exception: when a company has not yet operated then the player receives the
value one step below the current stock value. The company's stock value is
reduced by one step per share sold. Exceptions:
\begin{itemize}
	\item The stock value of a company which has not yet operated does not
	change.
	\item Stock values can not drop below 50M.
	\item The stock value of the MHE is not affected by sales once it has
	bought a 5-train.
\end{itemize}

The stock marker of a company is always placed on the bottom of a stack after a
sale, even if it did not move.

Shares may not be sold during the first SR, the director of a railway company
that has not yet operated cannot sell below 20\% of this company. Otherwise a
director may sell all his shares as long as another players owns at least 20\%
of the company so he can become the new director. Ties for ownership are
resolved in turn order starting from the old director. Up to 80\% of shares may
be in the bank pool. Companies in receivership have 100\% of their shares in the
bank pool.

\subsection{Buying shares}
A player may buy shares in a company if they are available, he has not sold
shares in that company in this SR, and the company is not in receivership. The
player decides whether to buy from the bank pool or IPO if there are shares in
both. Shares in the MHE are always in the bank pool. When buying from the bank
pool the player pays the share value to the bank and receives the share. When
buying from the IPO there can be one of two cases:
\begin{itemize}
	\item The company has not yet operated. The money is paid to the bank and
	the player receives the share. When a company floats it is fully
	capitalized by the bank.
	\item If the company has operated, the player pays the stock price to the
	company and receives the share. These shares are the result of the company
	issuing new shares.
\end{itemize}

\subsection{Floating a company}
A railway company floats as soon as the third of its original is bought from the
IPO. The company is floated and receives 5 times its current stock value as
starting capital. The remaining two shares are placed in the bank pool.

\end{document}