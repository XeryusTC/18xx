\documentclass[a4paper,twocolumn]{article}

\usepackage[hidelinks]{hyperref}
\usepackage[all]{hypcap}

\title{1858 Summary}
\author{Designed by Ian D Wilson}
\date{Summary by Xeryus Stokkel}

\begin{document}

\maketitle

\section{Map}
\begin{itemize}
	\item The map includes the Iberian Peninsula, or Portugal and Spain.
	\item France is the only off-board area.
	\item There are two ports: I2 (Bilbao) and A16 (South of Lisbon).
\end{itemize}

\section{Game setup}
\begin{itemize}
	\item Lay the map and stock market open on the table in a central location.
	\item Place the shares certificates and trains on the indicated board
	spaces. The share certificates should be ordered so that the Director share
	is on top of its respective pile.
	\item Place the money next to the board. The bank size is P12,000.
	\item Place the track tiles by the board. At the beginning of the game only
	the yellow track tiles are can be build, but the others should be available
	for inspection.
	\item Place the station markers and stock price markers for all companies
	near the board as well.
	\item Place the Round Marker on the circle indicated with ``SR'' for Stock
	Round.
	\item Each player should have enough space for several company charters,
	his/her stock certificates and his/her money.
	\item Appoint one or more players to be the Banker, they are responsible for
	transactions with the Bank.
	\item Give each player an initial starting capital as indicated in
	\autoref{tbl:start}.
	\item Give one of the cards numbered 1 to 6 to each as the players (use as
	many as there are players). The players should change position around the
	table so that they are in ascending clockwise order. The player with the
	lowest numbered card takes the Priority Deal card. The numbered cards are
	then be returned to the box.
\end{itemize}

The certificate limit is also indicated in \autoref{tbl:start}. Each Private
company, the Director share of a Public company, and a regular share of a Public
company all count as a single certificate for this limit. A player may not own
more certificates than the certificate limit.

\begin{table}
	\centering
	\caption{Initial start capital and share limit.}
	\label{tbl:start}
	\begin{tabular}{l|r|r|r|r}
	No. Players & 3 & 4 & 5 & 6 \\ \hline
	Start capital & P500 & P375 & P300 & P250 \\
	Certificate limit & 21 & 16 & 13 & 11 \\
	\end{tabular}
\end{table}

\section{Corporate entities}
\subsection{Private companies}
\begin{itemize}
	\item There are 22 Private companies in total, 17 of these can be floated
	from the start of the game. The further 5 can be floated from phase 3. A
	full list can be found on page 3 of the game rules.
	\item Private companies have only one share that is owned by either a player
	or a Public company.
	
	When owned by a player:
	\begin{itemize}
		\item The certificate counts towards the owning player's certificate
		limit.
		\item The revenue generated is paid directly towards the owner at the
		start of an operating round.
	\end{itemize}
	When owned by a Public company:
	\begin{itemize}
		\item The income generated by the Private company is added to the
		revenue of the Public company when it operates.
	\end{itemize}
	
	\item Some Private companies have two revenue numbers on them. The first
	number is the revenue generated during Phase 2, the latter number is the
	revenue generated during Phase 3 and up.
	\item Private companies can not be bought or sold, they can only be acquired
	through an auction.
	\item Private companies are closed at the end of the OR when Phase 5 begins.
	% TODO: ADD REFERENCE TO PRIVATE COMPANIES CLOSING
	\item All hexes on the map associated with a Private company are its ``home
	hexes'', these are reserved and free to build for the Private company.
\end{itemize}

\subsection{Public companies}
\begin{itemize}
	\item There are eight Public companies, they all operate in the same manner.
	\item Each Public company can have five or ten shares, one of these is a
	the Director share, this is a double share.
	\begin{itemize}
		\item In the case of five shares: the Director share is worth 40\% of
		the Public company, the remaining three shares are all worth 20\%.
		\item In the case of ten shares: the Director share is worth 20\% of
		the Public company, the remaining eight shares are all worth 10\%.
	\end{itemize}
	\item The player who holds the greatest number of shares in a company is its
	Director, he/she also holds the Director share.
	\item If a player who is not the Director owns a greater share in the Public
	company than its current Director he swaps two of his certificates in the
	company for the Director share and becomes the new Director.
	\item If two players own an equal amount of shares in a company then there
	is no change in Directorship.
	\item The Director makes all decisions on behalf of a company, he does not
	need to listen to other shareholders.
	\item When a Public company is closed it is available to be restarted later.
\end{itemize}

\section{Company closure}
If a Public company's stock marker falls below P50 then the following happens:
\begin{itemize}
	\item The company is immediately closed without any compensation.
	\item All the company's stocks are returned to the marked board spaces.
	\item The company's stock marker and all of its station tokens are removed
	from the board.
\end{itemize}

Private companies that have not yet been closed by the end of the OR that
started Phase 5 are closed then.
% TODO: ADD REFERENCE TO PRIVATE COMPANIES CLOSING

\section{Stock Rounds}
A Stock Round or ``SR'' consists of a series of turns, starting with the player
that holds the Priority Deal and continuing clockwise. The game begins with an
auction of the 17 Private Companies available at the start, this is then
followed by a SR.

In each SR the player may do the following:
\begin{enumerate}
	\item Sell: a player may do any of the following sell actions. He may
	determine the order in which to do these.
	\begin{itemize}
		\item Sell certificates
		\item Exchange Private companies for shares in Public companies.
		\item Convert 5-share Public companies to a 10-share Public companies.
	\end{itemize}
	\item Buy: a player may then perform one of the following buy actions.
	\begin{itemize}
		\item Buy one certificate of a Public company.
		\item Start a Public company.
		\item Auction a Private company.
	\end{itemize}
\end{enumerate}

When a player does neither of these actions they are deemed to ``pass''. The
SR ends when all players have passed consecutively, a player who takes an action
in a SR is therefore guaranteed to have another action.

\subsection{Selling stock}
\begin{itemize}
	\item To sell stock, the player transfers any number of share certificates
	to the Bank Pool, subject to these constraints:
	\begin{itemize}
		\item Private Companies may never be sold.
		\item Shares in a Public Company which has not yet completed a turn in
		an OR may not be sold.
		\item There may never be more than 50\% of shares of any Public company
		in the Bank Pool.
		\item The Director certificate of a company may never be in the Bank
		Pool. This means that there must always be a player owning at least 20\%
		of a Public Company.
	\end{itemize}
	The player receives from the Bank the current Stock price for each share
	sold.
	\item If a player is the Director of a company and he now owns less of that
	company than another player as a result of selling shares that other player
	becomes the new Director of the company. If there is more than one eligible
	player, the new Director is the player with the largest holding, or in case
	of a tie, the tying player closest to the left of the outgoing Director. The
	outgoing Director exchanges the Director certificate for two ordinary
	certificates and the new Director exchanges two ordinary certificates for
	the Director certificate. This exchange is done before resolving the rest of
	the sale.
	\item If the Director (the outgoing Director if a change occured) sells
	shares the Company's Stock price drops one space on the Stock market (note:
	one space in total, not one per share sold). If any other player than the
	Director sells shares the Stock price does not move. If the stock marker
	moves to a place that is already occupied by other stock markers then it is
	moved to the bottom of the stack. If the player has sold shares in more than
	one company of which he is the Director than he determines the order in
	which the stock markers drop.
	\item A player may exchange a Private company for a share in a Public
	company provided the following:
	\begin{itemize}
		\item The Director of the Public company agrees.
		\item The Public company is able to trace a route from one of its
		station markers to a home hex of the Private company, or have a station
		marker on a home hex.
	\end{itemize}
	The player takes a share and the Public company and the Public company
	receives the Private company in its treasury. Since this is a sell action a
	player may do this as often as he wants, but he may not go over the 60\%
	share limit.
	\item When a player exceeds his certificate limit he must sell as must stock
	as possible to no longer exceed the limit. This should be done at the first
	opportunity.
	\item As a sell action the player may convert a 5-share Public company to a
	10-share Public company. This is done by turning the remaining five shares
	on the company charter over. Certificates now represent half of their
	previous share, a Director share is worth 20\% and a regular share is worth
	10\%.
\end{itemize}

\subsection{Buying stock}\label{sec:buying}
Players may never buy stock from other players. All stocks are bought
from the bank or Public companies.

A player may do only one buy action on his/her turn. A player may not buy stock
or offer a Private company for auction when he/she is at the share limit.
\begin{itemize}
	\item To buy stock in a company a player must not own 60\% or more in that
	company, or have sold stock of that company earlier in the current SR. The
	player may take one certificate of the company and must pay the current
	stock for it. If the share comes from the bank pool then the price is paid
	to the bank, if it comes from the company charter then it is paid to the
	company's treasury. The share price of the company is never affected by this
	action. If the player now owns a larger share in the company than the
	current Director, he becomes the Director and two of his certificates for
	the Director share.
	\item Prior to Phase 5 the current player can pick a Private company to put
	up for auction, he/she must bid a multiple of P5 and at least equal to the
	Minimum Bid for the Private company. Players can bid on the Private company
	in clockwise order, increasing the bid by multiples of P5; or they can pass.
	If a player passes then he/she may not bid during this auction any more.
	The player with the highest bid after all other players have passed wins
	the action. The current player's turn is now over, regardless of whether
	he/she has won the auction.
	\item Prior to Phase 5 players can only start Public companies by exchanging
	a Private company. To do this pick an unstarted Public company and put its
	stock marker on the space on the stock market corresponding to the face
	value of the Private company, rounded down. If there is already a token on
	this space the stock marker is placed on the bottom of the pile. The player
	pays the current stock price into the company's treasury. The Private
	company is also placed into the Public company's treasury. The player
	receives the Director certificate and the remaining shares are put in the
	company treasury. Place 5 shares upside down in the treasury to denote that
	the company is a 5-share company. The player puts a station marker in one
	of the home hexes of the Private company. If there is no space for the
	station marker in one of these hexes then the conversion is not allowed.
	\item Starting from Phase 5 a player may start a Public company by
	purchasing the Director certificate. This is done by first setting an
	initial stock price by placing its stock marker in one of the red outlined
	spaces on the stock market (P70 to P150). If the space is already occupied
	the marker is put on the bottom of the stack. The cost is twice the stock
	price, paid into the Public company's treasury. The company must have enough
	money for for the station marker of the company may not be started. The
	current player puts one station marker of the Company on an unoccupied city
	circle anywhere on the map, and the Company immediately pays twice the value
	of the city to the bank (hexes without a tile or pre-printed value are
	free). The Public company must start as a 10-share company.
\end{itemize}

\subsection{Initial Stock Round}
At the start of the first stock round the initial 17 Private companies are
auctioned (see \autoref{sec:buying}) until all players pass. The minimum bid for
each private company is printed on the card. The winner of the auction pays the
bid to the Bank. Public companies may not be started in the first Stock Round.

\subsection{Ending the Stock Round}
The SR ends when all players pass consecutively. Nothing happens to the stock
value of fully sold out companies. The Priority Deal card is given to the player
left of the the last player to make a buy or sell action in the SR. If there
were no transactions in the SR then the Priority Deal does not move. The Round
Marker is moved to OR1.

\section{Operating Rounds}
There are two Operating Rounds (OR) between Stock Rounds. In each OR every
Private Company and started Public Companiy operate once. The player-owned
Private companies operate first in numerical order. They pay their income to
their owners and lay track. Next the Public Companies operate in order of
descending stock price. If stock price markers of two or more Public Companies
are on the same space on the stock market, the one on top operates first. Public
companies can execute several actions, some optional. They occur in this order:
\begin{enumerate}
	\item Optionally lay or upgrade one or two track tiles.
	\item Optionally place one station marker.
	\item Run train(s).
	\item Distribute earnings.
	\item Close Privates.
	\item Optionally buy trains.
	\item Optionally redeem or issue shares.
\end{enumerate}
All decisions on behalf of a Company are made by its Director, he does not have
to follow the advice of other shareholders. At the end of each OR the Round
Marker is moved to the next space, following the arrows.

\subsection{Track gauges}
There are three gauges of track available in the game: broad-gauge (solid black
line), metre-gauge (alternate black and white), and dual-gauge (white line
bordered with black) which is considred to be both broad and metre gauge.
M-trains may only run on metre-gauge or dual-gauge track. Other trains may only
run on broad-gauge or dual-gauge track. When tracing a route, a Company may only
change gauge at a city in which it has a station marker (mostly relevant for
building track). Metre-gauge track is not available until Phase 3.

\subsection{Routes}
A route of a Company is a continuous brod or narrow-gauge track segment,
including at least one city with a station marker belonging to the Company, or
in the case of a Private Company, including at least one home hex. It may not
reach or pass through any small or large city, port, or France more than once.
It may not use the same track segment more than once, except for the junctions
at the centres of four-way and five-way brown and grey plain track tiles. It may
not pass through any city completely filled with station markers belonging to
other Companies. If a route visits a port or France, it must terminate there.

Cities which have two or more separate disconnected stations, such as Madrid in
yellow, may be included in a route only once.

\subsection{Company actions}
\subsubsection{Lay or upgrade track}
\paragraph{General}
\begin{itemize}
	\item The active Company may, per Operating turn, lay a yellow tile, or
	upgrade one yellow tile or map hex to green, one green to brown, or one
	brown to grey when these are available.
	\item For an additional fee of P20, a Company may do another tile operation
	in a different hex; both operations may be an upgrade. If the new track on
	either of the two hexes is entirely metre-gauge then the fee is only P10.
	\item Green tiles are available from phase 3, brown from Phase 5, and grey
	from Phase 7.
	\item The supplied quantity of yellow track tiles (of all types) is intended
	to be sufficient for most games. if it runs out, more should be
	constructed. The tile mix of other colours is intended to limit play. If a
	vital tile is in play then it must first be upgraded in order to free it.
	\item When laying a yellow tile, it must be placed on and aligned with one
	of the pale green hexes on the map.
\end{itemize}

\paragraph{Private company}
\begin{itemize}
	\item While there is no track in a Private company's home hex, a Company may
	lay any suitable tile there following the normal building rules. Private
	companies may not lay a tile anywhere else until a tile has been placed	in
	at least one of their home hexes, preprinted track also fulfils this
	restriction. Otherwise, the track on the tile must form part of a route of
	the Private company.
	\item Track laid may not run of the hex grid or into the blank side of a
	red, blue or grey hex, nor may it run into an impassible hex-side shown by a
	thick blue line.
	\item If the map hex is marked with a large open circle, representing a
	large city, then so must the tile. If the map hex is also marked with a Y or
	B then so must the tile.
	\item If the map hex is marked with a dot representing a small city, then a
	tile with a bar must be used. If the hex contains two dots representing two
	small cities, a tile with two bars must be used; If this tile has two
	separate track sections, only one needs to form part of a route of the
	Company.
	\item If the map hex does not contain a dot or open circle then plain track
	(i.e. track without a city) must be used.
\end{itemize}

\paragraph{Terrain costs}
\begin{itemize}
	\item If the hex is labelled with a sum of money then the Company must
	immediately pay that sum to the bank. Terrain costs are halved when placing
	metre-gauge track. If the Company does not have sufficient funds, the
	Director may contribute cash from hand.
	\item A Private company does not pay terrain costs for its home hexes, but
	its player-owner must pay for any other terrain costs and extra builds. A
	Public company must pay for terrain costs, even if it owns the relevant
	Private Company.
	\item Only one of the indicated Private companies may lay yellow track on a
	home hex of a Private company until it closes, unless the Public company
	owns the relevant Private company.
	\item Private companies may lay track outside their home hexes, provided
	they can trace a route from a home hex.
	\item Private companies may not upgrade track.
	\item Private companies placing track in home hexes have to align the track
	to connect to the other home hexes of Privates shown in that hex.
\end{itemize}

\paragraph{Upgrading}
\begin{itemize}
	\item When upgrading track, the old tile is removed and becomes available
	for reuse, the new tile is substituted.
	\item Plain track, small city, normal city and B/L/M/Y city tiles must be
	replaced with corresponding tiles.
	\item Track segments on the old tile must be represented, in the same
	orientations, on the new tile.
	\item New track on the tile may not run off the hex grid or into the blank
	side of a blue, red, or grey hex, nor may it run into an impassable hex-side
	shown by a thick blue line.
	\item For plain and small city track upgrades, some part of the track on the
	new tile which is not on the old one must constitue an extension of a route
	of the Company. For other city upgrades, at least some part of the track
	(not necessarily a new part) on the new tile must be part of a route of the
	Company.
	\item If the tile has one or more station markers on it, those are kept on
	the new tile.
	\item There is no charge for upgrading track.
	\item The board comes preprinted with a number of tiles with yellow or green
	track. These hexes should be treated as though a yellow or green tile has
	been placed there already; yellow tiles may not be built there, but such
	hexes may be upgraded.
	\item When there is a choice of city upgrade available, the one with most
	exits which are legal must be laid.
	\item The brown P tile is reserved for Porto (B9).
\end{itemize}

\end{document}
