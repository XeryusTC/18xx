\documentclass[a4paper,twocolumn]{article}

\title{18C2C rules notes}
\author{}
\date{}

\begin{document}

\maketitle

\begin{itemize}
	\item Split share value companies always operate first
	\item Cannot go into lower stock market region (under the red line) unless 
	enough shares are sold to go two spaces below the line (even if those don't 
	exist).
	\item Stock value moves left on with hold/no dividend; moves right when 
	paying dividend/going cross continent; remains the same on 50/50.
	\item Can buy privates from each other.
	\item Company is sold out when no shares in IPO or pool, but there can be 
	shares in treasury.
	\item President of a company can buy shares from other player when holding 
	at least 50\% (see 4.3.1).
	\item A company may redeem (buy) shares in a stock round (on its presidents 
	turn):
	\begin{itemize}
		\item Can only do this if combined holdings of players and pool is 6 
		shares.
		\item Redeem from bank pool first, then from player's holdings.
		\item Can redeem from player's hand with their consent, does not count 
		as a sell action for that player.
		\item Must redeem from market first.
		\item May only redeem once per stock round.
		\item Must have operated before being able to redeem, can only redeem 
		after reissuing shares when it had enough money before reissue.
	\end{itemize}
	\item A company may reissue shares in a stock round when all its IPO shares 
	are sold (happens on its presidents turn):
	\begin{itemize}
		\item May do this only once per SR, but can do it more often.
		\item New value is $\max(\mbox{old par}, \frac{3}{4}\mbox{CMV})$.
		\item All redeemed shares are placed in the IPO.
		\item Company receives money from sale of the reissued shares, but 
		cannot use it until after the SR.
	\end{itemize}
	\item Full capitalization.
	\item Companies can merge at the end of the stock round.
	\item Track laying works with a point system, a company may use at most 4 
	points considering the following:
	\begin{itemize}
		\item Yellow track is 1 point, all other track is 2 points.
		\item May lay at most 3 points in the West/Plains/East zones, but 4 
		points in Canada.
		\item Companies starting in West/Plains may lay up to 4 points in their 
		first OR, but only if they are all within the same region.
		\item Merger companies, Amtrak and Conrail get 6 points, and they may 
		lay a bonus tile in Canada if they lay 3 tiles in Canada (for a total 
		of 4).
		\item Labor tokens give an additional point when all points are spend 
		in West/Plains.
		\item Merger companies may own two labor tokens, and the bonus of one 
		token is applied when building 3 points in the West/Plains.
	\end{itemize}
	\item Do not have to declare highest revenue if one run is the first
	continental run.
	\item IPO and redeemed shares pay to company.
	\item Merger companies must have a train in each shell to rise.
	\item May sell trains for \$1, Amtrak and Conrail may only buy at face 
	value.
	\item No train required if no route.
	\item Bankrupcy ends the game.
	\item Destinations work similar to 1822.
	\begin{itemize}
		\item Destination run only counts when ending in the destination, not
		when running through it. This is also true for the start token.
		\item Cannot have destination token and other token in the same hex,
		destination token replaces the regular one.
		\item Destination token only counts for one side of hexes with two
		stations.
		\item Destination tokens do not take up a token spot.
		\item Destination runs are extra run after the currently operating
		company, or first in the OR after connection happens in SR.
		\item When withholding destination run the token does not move.
		\item Merger companies do their destination run only with the shell that
		has reached the destination.
		\item Merger companies may shuffle trains before doing their run.
		\item Destination token counts double value after destination run.
	\end{itemize}
	\item Running from West Coast city to city east of the Mississippi gives
	a one extra move right on the stock market. Merger corporations can do this
	once for each shell.
	\item Merger companies may shuffle trains between shells at various moments
	in their turn.
	\item Merger companies may force a train buy by leaving a shell empty.
	\item Privates can be bought at any moment.
	\item Only one OR when the bank runs out of money.
	\item Some privates can be bought during phase 2 for up to face value.
\end{itemize}
	
\end{document}
