\documentclass[a4paper,twocolumn]{article}

\usepackage[a4paper]{fullpage}
\usepackage[compact]{titlesec}
\usepackage{enumitem}
\usepackage[hidelinks]{hyperref}
\usepackage[all]{hypcap}

\setlist{nosep}

\title{1858 differences from 18xx}
\author{Designed by Ian D Wilson\\
	Written by Xeryus Stokkel}
\date{}

\begin{document}

\maketitle

\section{Game setup}
\begin{itemize}
	\item Bank size is P12,000.
	\item Randomly determine the player ordering, reseat players so they are in
	clockwise order. The lowest number (or highest dice roll) becomes Priority
	Deal.
	\item The game starts with an auction of the Privates.
	\item Starting capital and share limits can be found in \autoref{tbl:start}.
\end{itemize}

\begin{table}
	\centering
	\caption{Initial start capital and share limit.}
	\label{tbl:start}
	\begin{tabular}{l|r|r|r|r}
		No. Players & 3 & 4 & 5 & 6 \\ \hline
		Start capital & P500 & P375 & P300 & P250 \\
		Certificate limit & 21 & 16 & 13 & 11 \\
	\end{tabular}
\end{table}

\section{Corporate entities}
\subsection{Private companies}
\begin{itemize}
	\item There are 17 Private companies (numbered 1 to 17) that are available
	from the start of the game.
	\item Five more Private companies become available in Phase 3.
	\item Each Private company only has one certificate, it is owned by one
	Player. It counts towards the certificate limit.
	\item Privates pay a fixed income, in Phase 2 this is the first revenue
	number, in later phases this is the second number. If there is only one
	number they always pay this.
	\item If owned by a player the Privates income is paid directly to their
	hand. If owned by a company it is added to its revenue when it operates.
	\item Privates may not be bought or sold.
	\item All Privates close at the end of the OR in which Phase 5 has started.
	\item Hexes on the map that are associated with the Private are its "home
	hexes", these are reserved and free to build for that Private. Company \#1
	has no board presence.
	\item Can lay track, but cannot own trains.
\end{itemize}

\subsection{Public companies}
\begin{itemize}
	\item Public companies can have 5 or 10 shares.
	\item If a Public company is closed it is available for restart later in the
	game.
	\item If a Company's stock marker falls below P50 then the company is closed
	and its shareholders are not compensated. Its trains are moved to the Bank
	Pool and its station markers are removed from the board.
	\item Can lay track, place station markers, operate and purchase trains,
	issue/redeem stock.
	\item A player may own at most 60\% of a Public company (exception: Private
	Closure Round, see \autoref{sec:closure}).
\end{itemize}

\section{Stock Rounds}
When a stock marker moves to a place on the stock market that is already
occupied by another marker then it is placed at the bottom of the stack.
\subsection{Sell actions}
A player may sell stock, exchange Privates for shares in Public companies, and
convert 5-share Publics to 10-share Publics. He may do this as often as he/she
wants
\begin{itemize}
	\item If a player who is the Director of a Public company sells enough stock
	than the player with the highest amount of stock closest to his left becomes
	the new Director.
	\item If the Director of a company (including the outgoing Director) sells
	shares, the company's value drops by one step regardless of the number of
	stocks sold. If a player who is not the Director sells stock the stock price
	does not move.
	\item The selling player always determines the order in which stock is sold.
	\item A player may exchange a Private for a share in a Public company from
	its Treasury. For this the Director of the Public company has to agree, and
	there must be a valid route from one of the Public company's station tokens
	to a home hex of the Private, or there is a station token on one of the home
	hexes. The Public company may place a station marker in one home hex free of
	charge, granted that there is space for it and a token is available.
	\item A player may not go over the 60\% share limit by exchanging shares.
	\item Exchanging shares can be done as often on a player's turn as he/she
	wishes.
	\item A player may change 5-share companies to 10-share companies if he is
	their Director. He does this by turning over the 5 remaining shares in the
	Treasury. Regular shares are now worth 10\% of the company, the Director
	share is now worth 20\%.
\end{itemize}

\subsection{Buy actions}
A player may do only one of the following buy actions on his/her turn:
\subsubsection{Buying stock}
This may not be done by a player at his/her certificate limit.
\begin{itemize}
	\item A player may only buy stock in a company if he owns less than 60\% of
	it. He may also not have sold its stock earlier in the current SR. Only one
	certificate may be bought at a time.
	\item If the certificate comes from the Bank Pool its price is paid to the
	Bank. If it comes from the Company Treasury the money is paid to the
	Treasury. The price is always the current stock value.
	\item Buying stock \textbf{never} affects the stock price.
\end{itemize}

\subsubsection{Auction private}\label{sec:auction}
This may not be done by a player at his/her certificate limit.
\begin{itemize}
	\item Before Phase 5 a player may elect an unsold Private company for
	auction. He/she must bid a multiple of P5 and at least equal to the Minimum
	Bid on the certificate.
	\item Players may raise the current bid by multiples of P5, going clockwise
	around the table. A player may also pass, he/she is then not allowed to bid
	in this auction anymore.
	\item When only one player is left bidding, he wins the auction and gains
	ownership of the Private. The winning bid is paid to the Bank.
	\item After the auction finishes it is the turn of the player who sits to
	the left of the player who initiated the auction.
\end{itemize}

\subsubsection{Start Public company}\label{sec:startpublic}
\begin{itemize}
	\item Before Phase 5 players may only start Public companies by exchanging a
	Private company. They place an unopened Public company's stock marker on the
	space on the stock market corresponding to the face value of the Private,
	rounded down. The player also pays this initial stock price into the Company
	Treasury. The Private company is also placed in the Public's Treasury. The
	company is started as a 5-share company; place 5 shares upside down in the
	Treasury to denote this. The player puts a station marker on a home hex of
	the Private that has space for it. If there is no space he/she must elect a
	different home hex, or the company may not start.
	\item Starting in Phase 5, players may start a Public company by purchasing
	the Director certificate by setting the initial stock price by placing the
	stock marker in the red bordered area (P70 to P150). The player pays twice
	the initial stock price into the company Treasury. The player puts a station
	marker on an unoccupied city circle on the map, the company pays twice the
	current value of the city to the Bank (a hex without a tile or pre-printed
	city value is free). The Company starts as a 10-share company.
\end{itemize}

\subsection{First Stock Round}
At the start of the first stock round the initial Private companies are
auctioned until all players consecutively pass on starting an auction.
Auctioning is done in the same way as in \autoref{sec:auction}. Public companies
may not be started in the first stock round.

\subsubsection{Ending the Stock Round}
\begin{itemize}
	\item The SR ends when all players pass consecutively.
	\item Nothing happens to the stock values of Public companies that have sold
	out.
	\item The Priority Deal card moves to the player who sits to the left of the
	last player who bought or sold something. If there were no transactions then
	the Priority Deal does not move.
\end{itemize}

\section{Operating rounds}
There are always two ORs between SRs. First the Private companies operate in
ascending numerical order, then the Public companies operate in descending order
of stock price.

\subsection{Routes}
A route consists of a continuous broad or narrow-gauge track segment with at
least one city with a station marker of the Company on it, in case of Private
companies it includes at least one home hex. It may not go through any small or
large city, port or France more than once. When it goes to a port or France it
must terminate there. It may not visit a track segment multiple times, with the
exception of the junctions at the centre of four-way or five-way brown or grey
plain tiles. It may not go through stations completely filled with tokens of
other companies. Tiles with several separate stations (like Madrid in yellow)
may only be included once per route.

\subsection{Gauges}
There are three track gauges: broad-gauge (solid black), narrow-gauge (dashed),
and dual-gauge (white with black border) which is considered to be both broad
and narrow-gauge. M-trains may only run on narrow and dual-gauge track, other
trains may only run on broad and dual-gauge track. When tracing a route a
company may only change gauge at stations at which it has a station token.
Narrow-gauge may not be build until Phase 3.

\subsection{Actions}
Public companies may execute actions in the following order, some are optional.

\subsubsection{Lay or Upgrade Track (optional)}
\begin{itemize}
	\item Companies may lay yellow track or upgrade track according to the
	regular permissive rules.
	\item For an additional fee of P20 a company may do a track operation on a
	different hex. Both operations may be an upgrade.
	\item If the new track on either of the two hexes is completely narrow-gauge
	then the fee is only P10.
	\item Yellow tiles (of all types) are considered to be unlimited.
	\item If there is no track in a Private company's home hex, a company may
	lay track there. Private companies may not lay track anywhere until there is
	a tile in one of their home hexes (preprinted track satisfies this
	condition).
	\item The track must form part of a route of the company.
	\item Track may not run into the blank side of a red, blue, or grey hex. Nor
	may it run into an impassable blue border.
	\item Companies must pay terrain costs when building track when the hex has
	a sum of money on it. Costs are halved when playing narrow-gauge track. The
	Director may pay cash from hand if the company has insufficient funds.
	\item Private companies do not pay terrain costs for their home hexes, but
	its owner must pay costs for extra lays and other terrain costs from his
	hand.
	\item Public companies must always pay for terrain costs, even if it owns
	the relevant Private company.
	\item Only one of the indicated Privates may lay yellow track on a home hex
	until it closes, unless the Public company owns the Private company.
	\item Private companies may lay track outside their home hexes given that
	they can trace a route there.
	\item Private companies may not upgrade track.
	\item Private companies placing track in their home hex must align it with
	the indicated route that connects other home hexes of the Privates shown on
	that hex.
	\item When upgrading a company must be able to trace a route to new track.
	This requirement is lifted for upgrading cities.
	\item There is no charge for upgrading.
	\item Some hexes have yellow or green track printed on them, these should be
	treated as if there is already a yellow or green tile on them. They may be
	upgraded like usual but no track may be placed there.
	\item When there is a choice of city upgrade, the one with the most legal
	exits must be picked.
	\item The brown P tile is reserved for Porto (B9)
\end{itemize}

\subsubsection{Place station marker (optional)}
\begin{itemize}
	\item A company may place one station marker per OR on a vacant space on a
	large city on one of its routes.
	\item A company may only have one station marker per hex.
	\item Each Public company has three station markers, those placed during
	formation are free (exception: see \autoref{sec:startpublic}).
	\item Prior to the Private Closure Round, one city space on each Private's
	home hexes is reserved for that Private. Public companies must leave that
	space free if the Private has not been closed yet. If there are two Privates
	then there are also two reserved spaces. If a Public company owns a Private
	company then it may ignore the restrictions for that Private company.
	\item The cost for placing a marker is P20 if the marker is in the same
	province as another station marker for the Public company. Otherwise the
	cost is P40 per provincial boundary crossed. The cost is determined from the
	closest station marker, in provinces, irrespective from the route involved.
\end{itemize}

\subsubsection{Run trains}
\begin{itemize}
	\item Trains must run different routes that must not share track segments
	with other trains. They may meet or cross at ports, large cities and France.
	\item The value of the route is the sum of the values of all visited ports,
	cities and off-board areas. Skipped cities do not count. The value of France
	corresponds to the current Phase number.
	\item The earnings of a company are the sum of the values of the routes run
	by its trains, plus the printed income of all Private companies that it
	owns.
	\item The highest legal income announced by any player must be declared, but
	players are not required to announce a higher income than the Director.
	\item There are four types of trains available:
	\begin{enumerate}
		\item H-trains run up to the given number in hexes on broad or
		dual-gauge track. Each hex border they cross counts toward this number.
		They count the value of all cities, ports and France they visit.
		\item M-trains run up to the indicated number of ports, large cities or
		France on narrow-gauge or dual-gauge. Small cities do not count towards
		the value of the train, but do count towards the value of the run. The
		route may start/end in a small city.
		\item E-trains run up to the indicated value of ports, large cities or
		France on broad or dual-gauge track. Small cities must be skipped, large
		cities may not.
		\item D-trains run like E-trains but double the value of the run.
	\end{enumerate}
	\item Wounded trains run as normal but the value of their run is halved.
\end{itemize}

\subsubsection{Distribute earnings}
\begin{itemize}
	\item Companies may withhold their earnings, pay in full, or pay half/half.
	\item When paying in full, divide the earnings by the number of shares (five
	or ten), round down to the nearest integer and pay that much per share to
	the shareholders. The remainder is paid to the company treasury.
	\item When paying half, halve the earnings before distributing them like
	full earnings. The remaining half goes into the Company Treasury.
	\item Payments for shares in the company's treasury go to the company,
	payments for shares in the Bank Pool remain in the Bank.
	\item If the company withholds its earnings, or its earnings are zero then
	its stock marker is moved one space to the left.
	\item If ten times the divident per share is equal to or greater than the
	current stock value then the stock marker moves one space to the right,
	unless it is already at the highest value. \emph{This means that 5-share
	companies need to pay only half of their stock value to move on the stock
	market, while 10-share companies need to pay their full stock price.}
	\item If a stock marker does not move then its relative position is
	maintained. If it does move then it always move to the bottom of a pile.
\end{itemize}

\subsubsection{Close Private companies}
All private companies owned by the Public company now close and are removed
from the game. The Public company receives their face values from the Bank into
its Treasury.

\subsubsection{Purchase trains (optional)}
\begin{itemize}
	\item A company may buy a train from another company with its Director's
	consent. The price paid must be at least P1 and at may not exceed the active
	Company's Treasury.
	\item A company may buy any train from the Bank Pool, or the newest train
	from the Bank (or, in Phase 7, a 5D). It must pay face value to the Bank.
	\item The company may buy multiple trains in the same OR if it has the space
	and resources. The effects of a purchase are applied as soon as the trains
	are bought.
	\item If a company is over its train limit it must discard trains to the
	Bank Pool without compensation until it is within the train limit. If more
	than one Company has excess trains they choose which to discard in operating
	order.
	\item Wounded trains count towards the train limit, but they may be
	discarded at will during this step.
	\item Discarded wounded trains are removed from the game, they do not end up
	in the Bank Pool.
	\item If the company has no train then it is not required to buy one, but if
	it elects not to buy one its stock value drops by one step.
	\item A company with a train may use Emergency Fund Raising to buy a train
	from the Bank or Bank Pool:
	\begin{enumerate}
		\item The Company may sell shares in its treasury to the Bank Pool, it
		receives the current stock price for each of them. The stock value drops
		by one step after this sale. The company may not sell more shares than
		it needs to buy a train. This may be done in a Company's first OR.
		\item The Director may put in cash from hand. He may not sell shares to
		raise more money. The Director may not put in more cash than the Company
		needs to buy a train.
	\end{enumerate}
	The company may buy any train from the Bank or Bank Pool, even if it has
	enough money for the cheapest train.
	\item If the company has cash left over after Emergency Fund Raising then it
	may use this money to buy a second train from another Company or the Bank
	according to the normal rules.
	\item If Emergency Fund Raising does not raise enough money for the company
	to buy a train, no money is raised and the Company does not buy a train.
	\item Some trains have a broad-gauge train on one side and a narrow-gauge
	train on the other side. When bought from the bank the buyer decides which
	way the train should go. This decision cannot be changed, even if the train
	is returned to the Bank Pool.
\end{itemize}

\subsubsection{Redeem and issue stock (optional)}
\begin{itemize}
	\item A company may buy back one or all of its shares in the Bank Pool and
	place them into its Treasury. It must pay the current stock value for each
	one bought.
	\item A company may sells shares in its Treasury to the Bank Pool. It
	receives the current stock price for each of them. This reduces the stock
	value by one step. A Company may not do this in its first operating turn.
\end{itemize}

\section{Private Closure Round}\label{sec:closure}
At the end of the OR that the first 5E/4M Train is bought, there is the Private
Closure Round. In numerical order, each player-owned Private Company must be
exchanged for a share in a Public Company, or be exchanged for cash from the
Bank.
\begin{itemize}
	\item When exchanging for a share, the Public company must be able to trace
	a route from one of its station markers to a home hex of the Private
	Company, or have a station marker on one of its home hexes. If there is a
	Treasury share and the Director gives permission then the player receives a
	Treasury share and the Company receives the face value of the private into
	its Treasury from the Bank. It may then place a station token on one of the
	Private's home hexes according to the normal rules.

	If there is no Treasury share or the Director does not give permission then
	the player may take a share from the Bank Pool if there are any. The Public
	Company does not receive any money. The Public Company must still have a
	route to one of the Private Company's home hexes.
	\item The player may receive face value of the Private in cash from the
	Bank.
\end{itemize}
In all cases the Private company is closed and removed from play. Unsold Private
Companies are also removed from play. A player may go over the normal 60\%
holding limit during this round, and he is not obliged to sell back to 60\% at
any point.

\section{Ending the game}
The game ends at the end of the next OR after the Bank runs out of money. If the
Bank runs out during an OR the game ends at the end of that OR. If the Bank runs
out of money during an SR then the SR continues as normal and one more OR is
played. If the Bank becomes solvent again, the game still ends as described.

A player's total wealth is the value of his stock at their current stock price
plus his cash in hand. Private companies are worth face value. The richest
player wins.

\end{document}
