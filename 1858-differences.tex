\documentclass[a4paper,twocolumn]{article}

\usepackage[hidelinks]{hyperref}
\usepackage[all]{hypcap}

\title{1858 differences from 18xx}
\author{Designed by Ian D Wilson\\
	Written by Xeryus Stokkel}
\date{}

\begin{document}

\maketitle

\section{Game setup}
\begin{itemize}
	\item Bank size is P12,000.
	\item Randomly determine the player ordering, reseat players so they are in
	clockwise order. The lowest number (or highest dice roll) becomes Priority
	Deal.
	\item The game starts with an auction of the Privates.
	\item Starting capital and share limits can be found in \autoref{tbl:start}.
\end{itemize}

\begin{table}
	\centering
	\caption{Initial start capital and share limit.}
	\label{tbl:start}
	\begin{tabular}{l|r|r|r|r}
		No. Players & 3 & 4 & 5 & 6 \\ \hline
		Start capital & P500 & P375 & P300 & P250 \\
		Certificate limit & 21 & 16 & 13 & 11 \\
	\end{tabular}
\end{table}

\section{Corporate entities}
\subsection{Private companies}
\begin{itemize}
	\item There are 17 Private companies (numbered 1 to 17) that are available
	from the start of the game.
	\item Five more Private companies become available in Phase 3.
	\item Each Private company only has one certificate, it is owned by one
	Player. It counts towards the certificate limit.
	\item Privates pay a fixed income, in Phase 2 this is the first revenue
	number, in later phases this is the second number. If there is only one
	number they always pay this.
	\item If owned by a player the Privates income is paid directly to their
	hand. If owned by a company it is added to its revenue when it operates.
	\item Privates may not be bought or sold.
	\item All Privates close at the end of the OR in which Phase 5 has started.
	\item Hexes on the map that are associated with the Private are its "home
	hexes", these are reserved and free to build for that Private. Company \#1
	has no board presence.
	\item Can lay track, but cannot own trains.
\end{itemize}

\subsection{Public companies}
\begin{itemize}
	\item Public companies can have 5 or 10 shares.
	\item If a Public company is closed it is available for restart later in the
	game.
	\item If a Company's stock marker falls below P50 then the company is closed
	and its shareholders are not compensated. Its trains are moved to the Bank
	Pool and its station markers are removed from the board.
	\item Can lay track, place station markers, operate and purchase trains,
	issue/redeem stock.
	\item A player may own at most 60\% of a Public company (exception: Private
	Closure Round, see \autoref{sec:closure}).
\end{itemize}

\section{Stock Rounds}
When a stock marker moves to a place on the stock market that is already
occupied by another marker then it is placed at the bottom of the stack.
\subsection{Sell actions}
A player may sell stock, exchange Privates for shares in Public companies, and
convert 5-share Publics to 10-share Publics. He may do this as often as he/she
wants
\begin{itemize}
	\item If a player who is the Director of a Public company sells enough stock
	than the player with the highest amount of stock closest to his left becomes
	the new Director.
	\item If the Director of a company (including the outgoing Director) sells
	shares, the company's value drops by one step regardless of the number of
	stocks sold. If a player who is not the Director sells stock the stock price
	does not move.
	\item The selling player always determines the order in which stock is sold.
	\item A player may exchange a Private for a share in a Public company from
	its Treasury. For this the Director of the Public company has to agree, and
	there must be a valid route from one of the Public company's station tokens
	to a home hex of the Private, or there is a station token on one of the home
	hexes. The Public company may place a station marker in one home hex free of
	charge, granted that there is space for it and a token is available.
	\item A player may not go over the 60\% share limit by exchanging shares.
	\item Exchanging shares can be done as often on a player's turn as he/she
	wishes.
	\item A player may change 5-share companies to 10-share companies if he is
	their Director. He does this by turning over the 5 remaining shares in the
	Treasury. Regular shares are now worth 10\% of the company, the Director
	share is now worth 20\%.
\end{itemize}

\subsection{Buy actions}
A player may do only one of the following buy actions on his/her turn:
\subsubsection{Buying stock}
This may not be done by a player at his/her certificate limit.
\begin{itemize}
	\item A player may only buy stock in a company if he owns less than 60\% of
	it. He may also not have sold its stock earlier in the current SR. Only one
	certificate may be bought at a time.
	\item If the certificate comes from the Bank Pool its price is paid to the
	Bank. If it comes from the Company Treasury the money is paid to the
	Treasury. The price is always the current stock value.
	\item Buying stock \textbf{never} affects the stock price.
\end{itemize}

\subsubsection{Auction private}\label{sec:auction}
This may not be done by a player at his/her certificate limit.
\begin{itemize}
	\item Before Phase 5 a player may elect an unsold Private company for
	auction. He/she must bid a multiple of P5 and at least equal to the Minimum
	Bid on the certificate.
	\item Players may raise the current bid by multiples of P5, going clockwise
	around the table. A player may also pass, he/she is then not allowed to bid
	in this auction anymore.
	\item When only one player is left bidding, he wins the auction and gains
	ownership of the Private. The winning bid is paid to the Bank.
	\item After the auction finishes it is the turn of the player who sits to
	the left of the player who initiated the auction.
\end{itemize}

\subsubsection{Start Public company}
\begin{itemize}
	\item Before Phase 5 players may only start Public companies by exchanging a
	Private company. They place an unopened Public company's stock marker on the
	space on the stock market corresponding to the face value of the Private,
	rounded down. The player also pays this initial stock price into the Company
	Treasury. The Private company is also placed in the Public's Treasury. The
	company is started as a 5-share company; place 5 shares upside down in the
	Treasury to denote this. The player puts a station marker on a home hex of
	the Private that has space for it. If there is no space he/she must elect a
	different home hex, or the company may not start.
	\item Starting in Phase 5, players may start a Public company by purchasing
	the Director certificate by setting the initial stock price by placing the
	stock marker in the red bordered area (P70 to P150). The player pays twice
	the initial stock price into the company Treasury. The player puts a station
	marker on an unoccupied city circle on the map, the company pays twice the
	current value of the city to the Bank (a hex without a tile or pre-printed
	city value is free). The Company starts as a 10-share company.
\end{itemize}

\subsection{First Stock Round}
At the start of the first stock round the initial Private companies are
auctioned until all players consecutively pass on starting an auction.
Auctioning is done in the same way as in \autoref{sec:auction}. Public companies
may not be started in the first stock round.

\subsubsection{Ending the Stock Round}
\begin{itemize}
	\item The SR ends when all players pass consecutively.
	\item Nothing happens to the stock values of Public companies that have sold
	out.
	\item The Priority Deal card moves to the player who sits to the left of the
	last player who bought or sold something. If there were no transactions then
	the Priority Deal does not move.
	\item The round marker is moved to OR1.
\end{itemize}

\end{document}